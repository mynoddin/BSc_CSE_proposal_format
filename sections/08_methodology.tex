%% ============================================================
%% 8. PROPOSED METHODOLOGY
%% ============================================================

\newpage
\section{Proposed Methodology}

\placeholder{This is the most detailed and critical section of your proposal. Describe your complete research methodology step by step. A reader should be able to replicate your approach from this description.}

%% ----- 8.1 Research Approach -----
\subsection{Research Approach}

\subsubsection{Philosophical Worldview}
\placeholder{Describe the philosophical stance underlying your research:}
\begin{itemize}[leftmargin=1cm]
    \item \placeholder{\textbf{Positivist} --- objective, measurable, quantitative data}
    \item \placeholder{\textbf{Interpretivist} --- subjective, qualitative understanding}
    \item \placeholder{\textbf{Pragmatic} --- mixed methods, practical outcomes}
\end{itemize}

% ===== WRITE YOUR PHILOSOPHICAL WORLDVIEW HERE =====


\subsubsection{Research Design Selection}
\placeholder{Specify your research design and justify your choice:}
\begin{itemize}[leftmargin=1cm]
    \item \placeholder{Experimental / Quasi-experimental}
    \item \placeholder{Survey-based / Questionnaire-based}
    \item \placeholder{Case study / Observational}
    \item \placeholder{Computational / Simulation-based}
\end{itemize}

% ===== WRITE YOUR RESEARCH DESIGN HERE =====


%% ----- 8.2 Data Collection -----
\subsection{Data Collection}

\subsubsection{Data Sources}
\placeholder{Describe all data sources with specifics:}
\begin{itemize}[leftmargin=1cm]
    \item \placeholder{Existing datasets (name, source URL, size, format)}
    \item \placeholder{Primary data collection (surveys, interviews, sensors)}
    \item \placeholder{Web scraping (target websites, data fields)}
    \item \placeholder{Synthetic data generation (if applicable)}
\end{itemize}

% ===== WRITE YOUR DATA SOURCES HERE =====


\subsubsection{Data Collection Procedures}
\placeholder{Explain the step-by-step procedure for collecting data. Include sampling strategy (random, stratified, convenience), tools used, and timeline for data collection.}

% ===== WRITE YOUR PROCEDURES HERE =====


\subsubsection{Data Validation}
\placeholder{Describe how you will ensure data quality, reliability, and validity. Include cross-validation strategies, inter-rater reliability (for manual labeling), and data cleaning verification steps.}

% ===== WRITE YOUR VALIDATION APPROACH HERE =====


%% ----- 8.3 Data Preprocessing -----
\subsection{Data Preprocessing}

\placeholder{Detail the preprocessing pipeline. Include all steps with justification:}
\begin{enumerate}[leftmargin=1cm, itemsep=4pt]
    \item \placeholder{Data cleaning (handling missing values, removing duplicates)}
    \item \placeholder{Normalization / Standardization}
    \item \placeholder{Feature extraction / Feature engineering}
    \item \placeholder{Encoding (one-hot, label, ordinal)}
    \item \placeholder{Data augmentation (if applicable)}
    \item \placeholder{Train-test-validation split ratios}
    \item \placeholder{Class balancing (SMOTE, undersampling, etc.)}
\end{enumerate}

% ===== WRITE YOUR PREPROCESSING STEPS HERE =====


%% ----- 8.4 Proposed Model -----
\subsection{Proposed Model / Framework / Algorithm}

\placeholder{Describe the core technical approach in detail. This is the heart of your methodology. Include:}
\begin{itemize}[leftmargin=1cm]
    \item \placeholder{Architecture diagram (see Figure~\ref{fig:architecture})}
    \item \placeholder{Algorithmic steps (see Algorithm~\ref{alg:proposed})}
    \item \placeholder{Mathematical formulations}
    \item \placeholder{System design and data flow}
\end{itemize}

% ===== SYSTEM ARCHITECTURE FIGURE =====
% Uncomment and replace with your diagram:
%
% \begin{figure}[H]
%     \centering
%     \includegraphics[width=0.85\textwidth]{figures/architecture.png}
%     \caption{Proposed System Architecture}
%     \label{fig:architecture}
% \end{figure}

% ===== ALGORITHM PSEUDOCODE =====
% Uncomment and modify for your algorithm:
%
% \begin{algorithm}[H]
% \caption{Proposed Algorithm}
% \label{alg:proposed}
% \begin{algorithmic}[1]
%     \Require Dataset $D = \{(x_i, y_i)\}_{i=1}^{N}$
%     \Ensure Trained model $M$
%     \State $D' \gets \text{Preprocess}(D)$
%     \State $D_{train}, D_{test} \gets \text{Split}(D', \text{ratio}=0.8)$
%     \State Initialize model $M$ with parameters $\theta$
%     \For{epoch $e = 1$ to $E$}
%         \For{mini-batch $B \subset D_{train}$}
%             \State $\mathcal{L} \gets \text{Loss}(M(B), y_B)$
%             \State $\theta \gets \theta - \alpha \nabla_\theta \mathcal{L}$
%         \EndFor
%         \State $\text{acc} \gets \text{Evaluate}(M, D_{test})$
%     \EndFor
%     \State \Return $M$
% \end{algorithmic}
% \end{algorithm}

% ===== WRITE YOUR MODEL DESCRIPTION HERE =====


%% ----- 8.5 Evaluation Metrics -----
\subsection{Evaluation Metrics}

\placeholder{Specify the metrics you will use to evaluate your approach. Choose metrics appropriate to your task:}

\begin{table}[H]
\centering
\caption{Evaluation Metrics}
\label{tab:metrics}
\renewcommand{\arraystretch}{1.4}
\begin{tabularx}{\textwidth}{|l|X|l|}
\hline
\rowcolor{headerblue}
\textbf{Metric} & \textbf{Description} & \textbf{Task Type} \\
\hline
Accuracy & Overall correct predictions / total predictions & Classification \\
\hline
Precision & True positives / (True positives + False positives) & Classification \\
\hline
Recall & True positives / (True positives + False negatives) & Classification \\
\hline
F1-Score & Harmonic mean of Precision and Recall & Classification \\
\hline
AUC-ROC & Area under the ROC curve & Classification \\
\hline
RMSE & Root Mean Squared Error & Regression \\
\hline
MAE & Mean Absolute Error & Regression \\
\hline
Execution Time & Training and inference time & All \\
\hline
\end{tabularx}
\end{table}

\placeholder{Justify why these specific metrics are appropriate for your research.}

% ===== SELECT AND JUSTIFY YOUR METRICS HERE =====


%% ----- 8.6 Tools and Technologies -----
\subsection{Tools and Technologies}

\begin{table}[H]
\centering
\caption{Tools and Technologies}
\label{tab:tools}
\renewcommand{\arraystretch}{1.4}
\begin{tabularx}{\textwidth}{|l|X|}
\hline
\rowcolor{headerblue}
\textbf{Category} & \textbf{Tools} \\
\hline
Programming Language & Python 3.x \\
\hline
ML/DL Framework & TensorFlow / PyTorch / scikit-learn \\
\hline
Data Processing & Pandas, NumPy \\
\hline
Visualization & Matplotlib, Seaborn, Plotly \\
\hline
Development Environment & Jupyter Notebook / Google Colab / VS Code \\
\hline
Version Control & Git, GitHub \\
\hline
Document Preparation & \LaTeX\ (Overleaf) \\
\hline
\end{tabularx}
\end{table}

% ===== MODIFY THE TABLE ABOVE WITH YOUR ACTUAL TOOLS =====


%% ----- 8.7 Ethical Considerations -----
\subsection{Ethical Considerations}

\placeholder{Discuss ethical aspects of your research. Address as applicable:}
\begin{itemize}[leftmargin=1cm]
    \item \placeholder{Ethical guidelines and compliance}
    \item \placeholder{Informed consent for data collection from human subjects}
    \item \placeholder{Data privacy and anonymization}
    \item \placeholder{Cultural sensitivity (especially for research involving communities)}
    \item \placeholder{Bias and fairness in ML models}
    \item \placeholder{Responsible use of AI}
\end{itemize}

% ===== WRITE YOUR ETHICAL CONSIDERATIONS HERE =====
