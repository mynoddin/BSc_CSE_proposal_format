%% ============================================================
%% 7. RESEARCH GAP
%% ============================================================

\newpage
\section{Research Gap}

\placeholder{Based on the literature review, clearly articulate the specific gaps in existing research that your study aims to fill. This section directly connects your literature review to your proposed methodology.}

\subsection{Identified Gaps}

\placeholder{List and explain specific gaps found in existing research. Be precise about what is missing or inadequately addressed. Examples of common gaps:}
\begin{itemize}[leftmargin=1cm]
    \item \placeholder{No existing study addresses [specific problem] in [specific context]}
    \item \placeholder{Previous methods achieve low accuracy on [specific task/dataset]}
    \item \placeholder{Existing approaches do not consider [specific factor/variable]}
    \item \placeholder{No publicly available dataset exists for [specific domain]}
\end{itemize}

% ===== WRITE YOUR IDENTIFIED GAPS HERE =====


\subsection{Novelty of Proposed Research}

\placeholder{Explain how your proposed research addresses these gaps and what new contributions it will make to the field. Be specific about what is novel --- a new algorithm, a new dataset, a new application domain, a new combination of techniques, etc.}

% ===== WRITE YOUR NOVELTY HERE =====


\subsection{Practical Significance of Filling the Gap}

\placeholder{Describe the practical implications of filling the identified research gap. Who benefits? How does it improve existing systems, processes, or understanding?}

% ===== WRITE PRACTICAL SIGNIFICANCE HERE =====
