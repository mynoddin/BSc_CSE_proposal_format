%% ============================================================
%% 2. PROBLEM STATEMENT
%% ============================================================

\newpage
\section{Problem Statement}

\placeholder{Clearly and precisely define the problem your research aims to solve. This is one of the most critical sections of the proposal --- it justifies the entire research.}

\subsection{The Challenge}

\placeholder{What is the core technical or scientific challenge? Define it specifically and concisely. Avoid vague descriptions --- be precise about what needs to be solved.}

% ===== WRITE THE CHALLENGE HERE =====


\subsection{The Importance}

\placeholder{Why is solving this problem important? What are the consequences of not addressing it? Quantify the impact if possible (e.g., ``X\% of patients are misdiagnosed...'').}

% ===== WRITE THE IMPORTANCE HERE =====


\subsection{Contributing Factors}

\placeholder{Discuss the internal and external factors that contribute to or complicate this problem. These could include technological limitations, resource constraints, social factors, or data availability issues.}

% ===== WRITE CONTRIBUTING FACTORS HERE =====


\subsection{Limitations of Previous Work}

\placeholder{What limitations exist in current solutions or approaches? Why are they insufficient? Be specific about what prior work has attempted and where it falls short.}

% ===== WRITE LIMITATIONS HERE =====


\subsection{Socio-Economic Context}

\placeholder{If applicable, describe the socio-economic context and real-world relevance of the problem. This is especially important for research involving communities, healthcare, agriculture, or public policy.}

% ===== WRITE SOCIO-ECONOMIC CONTEXT HERE =====
